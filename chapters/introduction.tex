\chapter*{Introducere} 
\addcontentsline{toc}{chapter}{Introducere}

Învățarea adaptivă este o abordare care utilizează tehnologia și metodologii pentru a adapta procesul de învățare în funcție de nevoile, ritmul și preferințele fiecărui individ. Conceptul 
central al teoriei este că învățarea personalizată și adaptivă poate crește implicarea și eficacitatea învățării. În același timp, flexibilitatea, feedback-ul în timp util și utilizarea 
datelor sunt esențiale pentru ajustarea strategiilor de învățare. [1]

Într-o eră ce se caracterizează prin contiuă dezvoltare și evoluție, metodele tradi\-țio\-na\-le de învățare adesea nu reușesc să satisfacă eficient nevoile tuturor cursanților. De exemplu, în 
sălile de clasă mari este dificil să se adapteze materialul lecțiilor în ritmul de învățare al fiecărui student. Acest lucru evidențiază necesitatea unei abordări mai adaptive și personalizate 
pentru a îmbunătăți rezultatele învățării.[1]

Elemtele cheie ce trebuie urmate pentru a crea un mediu de învățare adaptiv sunt: flexibilitatea, feedback-ul și utilizarea datelor pentru ajustarea strategiilor de învățare și 
identificarea nevoilor și preferințelor de învațarea ale fiecărui student. În același timp, dezvoltarea de conținut și strategii de învățare care pot fi ajustate dinamic pe baza datelor și 
a feedback-ului obținut în timpul procesului de învățare este esențială. Totodată, este necesară implementarea de tehnologii care sprijină învățarea adapativă, cum ar fi platformele de
e-learning care utilizează algoritmi pentru a adapta materialele lecțiilor, iar toți acești factori trebuie susținuți de o curriculă flexibilă ce permite satisfacerea nevoilor 
studenților. [1] 

Această lucrare propune dezvoltarea unei extensii pentru platforma Moodle care integrează tehnologii de procesare a limbajului natural și de generare automată de conținut educațional. 
Prin încarcarea unui document în format PDF, extensia va extrage înformațiile și pe baza acestora va structura materialul în module de învățare, iar pentru fiecare modul va genera teste. 
Accesul studentului la modulele ulterioare este condiționat de finalizarea și promovarea testului de la modulul curent, implementând astfel un parcurs educațional secvențial și adaptiv. 
Scopul principal este de a îmbunătăți rezultatele educaționale și a experienței de învățare a studenților, reducând astfel decalajul de performanță prin personalizare. Totodată, ajustarea 
ritmului de învățare și evaluarea contiună a progresului și a performanței sunt trăsături esențiale ale învățării adaptive, iar extensia propusă permite acest lucru prin testele generate
automat după finalizarea fiecărui modul.[1] 