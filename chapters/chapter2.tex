\chapter{Tehnologiile Utilizate}

\section{Moodle}

Moodle este sistemul de management al învățării (LMS) pentru care este concepută această extensie. Moodle este o aplicație web scrisă în PHP [3], de tip open-source, care are ca scop 
oferirea unui mediu unificat de învățare profesorilor și studenților. Aceasta a luat naștere dintr-un proiect de cercetare doctorală (PhD) condus de Martin Dougiamas, cu ajutorul lui 
Peter C. Taylor, la Curtin University of Technology. Scopul principal al proiectului de cercetare a fost de a explora cum software-ul de internet poate sprijini cu succes epistemologii 
de predare și învățare bazate pe construcționismul social. Întrebarea principală a cercetării a fost clară: ce tipuri de structuri web și interfețe ajută sau, dimpotrivă, împiedică 
implicarea activă a participanților într-un dialog reflexiv, într-o comunitate de învățare? Accentul a fost pus pe sprijinirea lecturii deschise, a reflecției critice și a unei scrieri 
constructive, care să pornească din experiențele personale ale cursanților. Pe baza acestor observații, dezvoltarea platformei Moodle este orientată constant de această analiză, 
fiind gândită ca un instrument care sprijină și îmbunătățește procesele de învățare reflexivă în comunitate. [8]

Acest model open-source permite ca modificările făcute de utilizatori să fie adesea integrate în proiectul principal, permițând 
software-ului să evolueze conform valorilor comunității de utilizatori. Designul Moodle a fost gândit specific pentru a fi compatibil, flexibil și ușor de modificat. 
Este scris în limbajul PHP și este construit modular, utilizând tehnologii comune. Această abordare modulară a fost adoptată inițial pentru a permite modificarea 
rapidă a interfețelor ca răspuns la analiză și interese de cercetare, dar acum permite și altor programatori să modifice și să extindă codul. [8] Adaptibilitatea este o caracteristică
importantă pentru cei ce aleg să folosească Moodle. Spre deosebire de software-ul restricționat de licențe care limitează personalizarea, Moodle permite accesul la codul sursă și 
modificarea acestuia, această flexibilitate fiind văzută ca un mare avantaj. Capacitatea de a personaliza Moodle a fost un motiv pentru adoptare la instituții precum Otago Polytechnic 
din Noua Zeelandă și  Dublin City University (DCU).[9]

Distribuția Moodle are trei componente principale: codul sursă ce rulează pe un server web, o baza de date relațională (MariaDB) destinată stocării datelor și un spațiu de depozitare pentru
toate fișierele folsoite (folderul moodledata). Cursurile, activitățile și resursele sunt stocate în baza de date, iar extensiile instalate sunt stocate în folderul moodledata. Extensia 
creată intră în această categorie ca un modul de activitate. Moodle definește un modul de activitate ca fiind o extensie în care studentul interacționează cu alți studenți sau cu 
profesorul. Într-o activitate, studenții pot contribui direct la ceva propus de profesor, cum ar fi o resursă în forma unui fișier sau a unei pagini web.[3] Extensia este formată dintr-un
set de fișiere PHP, precum version.php, lib.php, view.php,  etc. și fișiere CSS și JavaScript pentru stilizare și interactivitate. Moodle apelează funcțiile extensiei în momentul în care
utilizatorul accesează pagina acesteia în interfața web. Prin utilizarea API-ului Moodle, extensia are parte de mecanisme standard pentru gestionarea bazei de date, controlul accesului
și a permisiunilor și de integrare, instalare și actualizare ușoară.

\section{XAMPP}

Pentru dezvoltarea, testarea locală a extensiei și pentru rularea Moodle, a fost utilizat XAMPP. XAMPP este un pachet software gratuit care conține Apache, MariaDB, PHP și 
Perl, „XAMPP este cel mai popular mediu de dezvoltare PHP”[4]. În contextul dezvoltării extensiei, Apache servește ca server web pentru aplicația Moodle și răspunde la cererile HTTP ale
utilizatorilor. Am ales XAMPP deoarece facilitează o configurare rapidă și simplă a mediului de dezvoltare deoarece nu este necesară instalarea separată a fiecărei componentă în parte.

\section{MariaDB}

MariaDB este sistemul de baze de date relațional folosit pentru stocarea datelor aplicației Moodle. MariaDB a fost creat inițial ca „fork al MySQL”, menținând compatibilitatea cu 
acesta la nivel de protocol și dialect SQL[4]. Principalele motive pentru alegerea MariaDB ca sistem de baze de date pentru Moodle sunt:
\begin{itemize}
    \item \textbf{Performanță și scalabilitate}: MariaDB este recunoscută pentru performanța sa îmbunatățită și pentru facilitarea unei scalări mai eficiente a aplicațiilor, 
    ceea ce este esențial pentru Moodle, având în vedere numărul mare de utilizatori și volumele de date gestionate. Teste comparative au demonstrat că MariaDB depășesțe MySQL în
    special în mediile cloud, permițând un debit mai mare de tranzacții și o latență mai mică în accesarea datelor pentru fire de execuție multiple ce rulează în paralel. [11]
    \item \textbf{Motoare de stocare suplimentare}: Pe lângă motoarele de stocare de date din MySQL, MariaDB oferă și altele noi, cum ar fi XtraDB, MariaDB ColumnStore și Aria,
    care îmbunătățesc performanța și flexibilitatea în gestionarea datelor. Aceste motoare permit optimizarea stocării și accesului la date, ceea ce este benefic pentru Moodle,
    având în vedere diversitatea tipurilor de date și a volumelor de informații gestionate. [11]
    \item \textbf{Caracteristici avansate pentru medii cu sarcini ridicate}: MariaDB oferă posibilitatea de a gestiona peste 200.000 de utilizatori conectați simultan. În același timp, 
    aceasta are capacitatea de a cripta jurnalele binare și tabelele temporare. [11]
    \item \textbf{Suport pentru Clustering și toleranța la erori}: Un atribut important al MariaDB este capacitatea de a crea clustere de baze de date care permit gestionarea distribuției 
    sarcinilor pe nodurile serverului de baze de date. Utilizarea acestei funcționalități îmbunătățește toleranța la erori, asigurând o redistribuire a traficului în caz de eșec al unui 
    nod. Studiile arată că utlizarea acestei caracteristici poare îmbunătăți seminificatic perfomanța mediului Moodle. [11]
    \item \textbf{Optimizare pentru configurații specifice}: Pentru sistemele bazate pe Windows, MariaDB este considerată configurația optimă pentru instalarea Moodle. Această variantă 
    oferă avantaje de clustering al bazei de date și echilibrare a sarcinii, asigurând calitate optimă, viteză ridicată și absența problemelor de securitate. Deși Apache/MySQL este 
    considerată optimă pentru distribuțiile Linux, MariaDB are potențialul de a înlocui MySQL în acest context. [11]
\end{itemize}

\section{GeminiAPI}

Gemini face parte din familia de modele generative de la Google DeepMind/Vertex AI. Aceste modele de inteligență artificială pot procesa intrări multimodale, text și imagini, și generează
conținut corent. Gemini este „cel mai capabil și general model AI” creat de Google până în prezent, optimizat în variante precum Ultra, Pro și Nano pentru sarcini de complexități diferite. 
Gemini este accesil prin API-ul Google Cloud Vertex AI, care permite aplicațiilor externe să trimită prompt-uri și să primească răspunsuri textuale generate. [5]

Domeniul Educational este unul dintre cele mai importante domenii de aplicare pentru inteligența artificială, iar aceasta are datoria de a transforma modul în care se predă și se învață, 
elevii numărându-se printre cei mai entuziaști utilizatorimai intrumentelor de intelingență artificială generativă. Deși nu există un reper robust și general pentru a evalua modelele AI 
pentru învățare, evaluările existente se concentrează adesea pe sarcini educaționale cum ar fi acuratețea la examene sau identificarea greșelilor. Predarea eficientă necesită mai 
mult decât suma acestor capacități individuale, necesită o înțelegere a cum și când sunt utilizate în practică. [13] 

\section{PHP}

Limbajul de programare folosit este PHP sau Hypertext Preprocessor. PHP este un limbaj de scriptare open-source, executat pe server, utilizat la scară largă pentru dezvoltarea de aplicații 
web dinamice. PHP poate genera conținut HTML dinamic, poate gestiona formulare web, poate interacționa cu sisteme de fișiere și poate efectua operații „Create, Read, Update, Delete” pe baze 
de date. De asemenea, acesta permite efectuarea de apeluri HTTP către API-uri externe, facilitând cuminicarea cu servicii precum Gemini API, folosit pentru generarea modulelor de învățare. [7]