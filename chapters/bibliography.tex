\chapter*{Bibliografie} 
\addcontentsline{toc}{chapter}{Bibliografie}

\begin{itemize}
    \item [1] D. B. A. Prof. Dr. Yoesoep Edhie Rachmad, \textit{Adaptive Learning Theory}, 2024, doi: 10.17605/OSF.IO/VFZ38.
    \item [2] P. Brusilovsky și E. Millán, \textit{User Models for Adaptive Hypermedia and Adaptive Educational Systems}, în The Adaptive Web, vol. 4321, P. Brusilovsky, A. Kobsa, și W. Nejdl, Ed., în Lecture Notes in Computer Science, vol. 4321. , Berlin, Heidelberg: Springer Berlin Heidelberg, 2007, pp. 3–53. doi: 10.1007/978-3-540-72079-9\_1.
    \item [3] \textit{Moodle architecture - MoodleDocs} [Online]. Disponibil la: https://docs.moodle.org\\/dev/Moodle\_architecture
    \item [4] \textit{MariaDB Knowledge Base}, MariaDB KnowledgeBase [Online]. Disponibil la: https://mariadb.com/kb/en/
    \item [5] \textit{Introducing Gemini: our largest and most capable AI model}, Google [Online]. Disponibil la: https://blog.google/technology/ai/google-gemini-ai/
    \item [6] \textit{Generate content with the Vertex AI Gemini API | Generative AI on Vertex AI}, Google Cloud [Online]. Disponibil la: https://cloud.google.com/vertex-ai/generative-ai/docs/model-reference/inference
    \item [7] \textit{PHP Introduction} [Online]. Disponibil la: https://www.w3schools.com/php/\\php\_intro.asp
    \item [8] M. Dougiamas și P. Taylor, \textit{Moodle: Using Learning Communities to Create an Open Source Course Management System}, prezentat la EdMedia + Innovate Learning, Association for the Advancement of Computing in Education (AACE), 2003, pp. 171–178 [Online]. Disponibil la: https://www.learntechlib.org/primary/p/13739/
    \item [9] E. Costello, \textit{Opening up to open source: looking at how Moodle was adopted in higher education}, Open Learning: The Journal of Open, Distance and e-Learning, vol. 28, nr. 3, pp. 187–200, nov. 2013, doi: 10.1080/02680513.2013.856289.
    \item [10] \textit{XAMPP Installers and Downloads for Apache Friends} [Online]. Disponibil la: https://www.apachefriends.org/
\end{itemize}