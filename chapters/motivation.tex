\chapter*{Motivație} 
\addcontentsline{toc}{chapter}{Motivație}

În contextul actual al sistemului educațional, nevoia de personalizare și adaptare a procesului de învățare este tot mai importantă. Metodele tradiționale de predare nu mai sunt suficiente 
și nu țin pasul cu diversitatea nevoilor unui student, diferitele stiluri de învățare și nu în ultimul rând evoluția tehnologiei. În același timp, sitemele existente de e-learning, deși oferă
acces facil la o gamă largă de resurse educaționale, nu reușesc întotdeauna să țină cont de particularitățile fiecărui student. Ca soluție pentru această problemă a apărut învățarea adaptivă,
un concept ce presupune ajustarea dinamică a conținutului și parcursului educațional în funcție de performanța și nevoile fiecărui student, aceasta devenind o soluție tot mai atractivă în 
mediul academic.[2]

Moodle, una dintre cele mai utilizate sisteme de gestionare a învățării la nivel global, oferă deja un cadru flexibil pentru dezvoltarea de extensii care să îmbunătățească întreg procesul educațional
pentru studenți, profesori și instituții. Cu toate acestea, crearea materialelor didactice rămâne un mare cosumator de timp pentru profesori, deoarece în continuare sunt realizate manual.
Astfel, generarea automată a conținutului educațional și a testelor de evaluare pe baza unor documente deja existente ar putea reudce semnificativ acest efort.

Motivația lucrării este dublă. În primul rând se urmărește imbunătățirea expe\-rien\-ței de învățare a studentului prin crearea unui mediu dinamic, care se adaptează la nivelul fiecăruia. În 
al doilea rând, se încearcă automatizarea și eficientizarea muncii profesorului, astfel progresul metodelor de predare este susținut de tehnologie și de inteligența artificială.

